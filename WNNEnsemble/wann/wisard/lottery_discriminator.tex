\documentclass{article}

\usepackage{amssymb}

\begin{document}
   \title{The Lottery Discriminator}
   \author{Douglas Cardoso}
   \date{\today}
   \maketitle

   \begin{abstract}
      Abstract here.
   \end{abstract}

   \section{Introduction}

      Let $D$ be a WiSARD discriminator composed of $|D|$ neurons. For a
      location $l$ in neuron $D_i$, how many times $l$ was written is
      represented as $D_i(l)$.  Consider that $D$ recorded $w$ inputs. An input
      $P$ is a vector of $|D|$ integers, each represented as $P_i$.

      Now consider that an input $P$ is used to query $D$. Let $H$ be the set
      of indexes of the neurons which positively answered (hits) to the query,
      and $M$ be the complement of $H$ (misses):
      
      $$H = \{\, i \, | \, D_i(P_i) > 0 \, \}, \quad M = D \setminus H.$$

      The answer $D$ gives to this query is

      $$
      A = \
      \overbrace{
         \left( \prod_{i \in H} \frac{D_i(P_i)}{1 + w} \right)^{|H|^{-1}}
      }^\alpha \
      \overbrace{\left( \frac{1}{1 + w} \right)^{|M|^k}}^\beta \
      $$

      This answer is an attempt to represent the original answer $D$ gives to a
      query as a scalar, translating a better answer in a greater value. That
      led to the definition of the two parts of the equation:

      \begin{itemize}

         \item $\alpha$ represents how good were the hits the discriminator
            made: it is the geometric mean of the rating of each hit. Each hit
            is rated according to $D_i(P_i)$. The innovation here is to make
            this score relative to $w$, what enables a better handling of
            unbalanced datasets by a WiSARD system;

         \item $\beta$ penalizes the answer according to the misses of the
            discriminator. The penalty increases with $w$, but its major
            influence is $|M|$. The parameter $k \in \mathbb{R}$ allows the
            adjustment of the importance of the misses in an answer.

      \end{itemize}

      Targeting an easier computation without giving way any property, $\log A$
      could be used:

      $$
      \log A = \
      \frac{1}{|H|}\left(\sum_{i \in H} \log D_i(P_i)\right) - \
      \log(1 + w) (1 + |M|^k) .
      $$

      Two interesting questions arise, targeting a better interpretation of an
      hypothetical raw value of $\log A = A'$. For the first question consider
      that all hits have the same grade $\delta$:

      $$ \forall\, i \in H, \quad \frac{D_i(P_i)}{1 + w} = \delta. $$

      Given $|M|$, what is this grade? It is

      $$ \delta = \exp\left( A' + \log(1 + w) |M|^k \right).$$

      The second question is: what is the maximum number of misses $\mu \in
      \mathbb{N}$ possible, considering that all hits were as good as possible:

      $$ \forall\, i \in H, \quad \frac{D_i(P_i)}{1 + w} \simeq 1. $$

      A good estimation of this is

      $$
      \mu = \
      \left\lfloor \left(\frac{A'}{-\log(1+w)}\right)^{1/k} \right\rfloor.
      $$

\end{document}
